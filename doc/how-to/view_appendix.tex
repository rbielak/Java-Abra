\chapter {Full listings of Switches and Options}

\section{XML Map file options}
These can also be found in codegen.dtd at http://charzard.


\# note include and relation are to be implemented ..

\begin{itemize}
\item{mapping ( class+,  property* )}
the top level element for all Abra files

\item{class ( description?, cache-type?, map-to?, class-view+, constraint+, field+ )}
\begin{description}
\item[name] the fully quaified class name - (ie. com.foo.Bar)
\item[extends] a Java class to extend (does not have to be in the map-file)
\item[identity] for SQL-mapping the primary key of this table
\item[implements] any Java interfaces to implement (; seperated) 
\item[factory-ext] the Java base factory for the generated Abstract fac defaults to (FactoryBase or EndDateFactoryBase (end-date=''true'')
\item[many-to-many] set to true for many to many relationships
\item[end-date] set to true if the class should be end-dated not deleted.
\end{description}

\item{constraint}
\begin{description}
     \item[name] 
     \item[type] 'unique' or 'check'
\end{description}
\item{relation ( description?, map-to?, related, related) } : NOT implemented
      name        
\item{related ( description?, sql?, xml?, ldap? )} : NOT implemented
\begin{description}
    \item[name]
    \item[type] 
\end{description}
\item{ map-to}
\begin{description}
    \item[table] the sql table name
    \item[xml] the xml node name
\end{description}

\item{class-view}: a definition of a class data-view (to abstract this
class)
\begin{description}
    \item[name]
    \item[format] a short name to be use later like 'normal', 'blotter'
\end{description}



\item{field ( description?, sql?, xml?, view+ , validate*)} : description of a java/sql/xml field mapping 
\begin{description}
    \item[name]
    \item[type] Fully qualified name (integer, string or foo.bar.FooBar)
    \item[required] if true both sql and xml will require this field
    \item[collection]    ( array | vector | hashtable | collection | set | map ) 
\end{description}

\item{sql}: for field to sql mappings
\begin{description}
    \item[name]
    \item[constraint-name] for referential or defined contraints
    \item[constraint] to define a field constraint (like '> 0')
    \item[type] to re-define the sql type (otherwise the default
mappings are used
\end{description}

\item{view}: to define how this field is viewed in different data abstractions
\begin{description}
    \item[in] the format of the view (like 'blotter')
    \item[foreign-view] if a composite field the name of the view in
the other class
	\item[foreign-field] if a composite field the single parameter to rip
\end{description}
 
\item{xml}: to define a how the java maps to xml
\begin{description}
    \item[name]
    \item[type]
    \item[node]  ( attribute | element | text ) how it should be (un)/marshalled
\end{description}
\item{validate}: add a validation node to this field
\begin{description}
	\item[required] set to true for business level mandatory field
	\item[regex] give a regular expression for pattern matching
	\item[name] used for regexp failed messages
	\item[code] used for regexp failed codes
\end{description}
\end{itemize}


\section{Switches in MapToJava}
The list of switches available in MapToJava. 

\begin{itemize}
\item Generation options
\begin{description}
\item{-outdir \textless directory\textgreater} : redifines the base output directory from . to
the given one

\item{-schema \textless schema-file\textgreater} : create a schema file of this name

\item{-factories} : create factories

\item{-facext \textless factory-name\textgreater} : if -factories is
set use this factory as base (unless a class explicitly declares
factory-ext=) (DEFAULTS to FactoryBase)
\item{-implements \textless class-name\textgreater} : use this class
as a global implements for all classes generated.
\item{-verify} : generate nothing just make sure the map is valid XML
\item{-validation} : generate Base Validators
\item{-procs} \textless pkg\textgreater : generate stored procedures and use them to store and update
\item{-xp} : use Abra Xml Parser instead of Xerces
\end{description}
\item Text Feedback options

\begin{description}
\item{-supress} : only print out Error messages
\item{-verbose} : a longer print out listing each class touched
\item{-mega} : a mega print out with all the verbose info and info
about each member field.
\end{description}

\item Props file
\begin{description}
\item {-props \textless props-file\textgreater} : instead of the above
command-line flags a props file can be specified with all the above
flags.
\item an example file
\begin{verbatim}
outdir=/home/paulb/Development/java
files=/home/paulb/Development/schema/test_map.xml
schema=../schema/new_merlin_schema.sql
factories=true
\end{verbatim}

\end{description}

\end{itemize}
\section{List of xml to Java/SQL mappings}


\begin{tabular}{|c|c|c|c|}
\hline
XML & Java & SQL & notes \\ \hline \hline
string & java.lang.String & varchar (n) & n is specified in the len
attribute \\ \hline

integer & int & int & - \\ \hline

date & java.sql.Timestamp & date & db uses GMT.. \\ \hline

long & long & number (38,0) & - \\ \hline

boolean & boolean & varchar (1) & the 'T'/'F' is stored and retrieved \\

float & float & float & - \\ \hline

double & double & double & - \\ \hline

longstring & java.lang.String & clob & when a string could be \textgreater then 4000 characters \\ \hline

\hline
\end{tabular}

note: to create a vector use the following syntax

\begin{verbatim}
<field name="children" type="com.foo.Person" collection="vector"/>
\end{verbatim}
And for an array use \textit{collection=''array''}
The type refers to the type of each element and the
collection="vector" makes this a vector.

