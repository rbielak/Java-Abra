\documentclass[10pt,openany]{book}


\title{Building Java Classes and a Database Persistence Layer Using
Code Generation - The Abra Project}
\author{Paul M. Bethe}
\date{\today}
\pagestyle{headings}
\begin{document}
\frontmatter
\maketitle
\chapter[Acknowledgments]{}
\section*{Thanks!}
The main programmers involved w/ The Abra project and its use:
\begin{description}
\item[Richie Bielak] \textless richieb@netlabs.net\textgreater
\item[Theodor Zagorin] \textless tzagorin@yahoo.com\textgreater
\item[Chris Marty] \textless cmarty@yahoo.com\textgreater
\item[Paul Bethe] \textless pmbethe@yahoo.com\textgreater
\end{description}

\chapter[Preface]{}
\section*{Preface}
Just a little write up of the Code generator tools first developed for the
TradingLinx Merlin system and later Open-Sourced onto
http://abra.sourceforge.net.  Chapters: 
\begin{description}
\item[Chapter 1] A How-To for learning and using the code generation.

\item[Chapter 2] For those interested in the conceptual grounding and
history of the Abra project.

\item[Chapter 3] For Developers: the core Abra classes and the
framework of the database and tools packages 

\item[Appendices] All the options for command-line runs, and elements
that can be used in XML-files 
\end{description} 

\nopagebreak
\tableofcontents

\mainmatter

\chapter{A Code Generation How-To}
\section{Intro}
This is a quick walk through use of the code generator from a basic
file to database schema and factories, adding database constraints and
finally creating abstracted data views of classes.


Also commonly referred to as Dave Knull\footnote{a bad joke that started from \texttt{/dev/null}}
the code generator is growing to be a very powerful tool, hopefully
all of the features will be understood after this chapter.
Note: this paper assumes you have a general knowledge of
XML\footnote{See \textit{XML Pocket Reference} by Robert Eckstein} and
Java\footnote{Try \textit{Elements of Java for the Principled Programmer} Bailey,
Duane.}.

\section{In the Beginning}
This tutorial will walk you through all the pieces of an XML map-file
that you can add, using the class User as an example.

\subsection{MapToJava aka Dave Knull}
All code generation starts with one class \texttt{org.ephman.abra.tools.MapToJava}.
First you need the entire package compiled then you can
run
\begin{description}
\item \texttt{java org.ephman.abra.tools.MapToJava \textless
your-map-file\textgreater}\footnote{For a complete listing of the flags
see Appendix A}
\end{description}
This file can be run on XML documents conforming to the DTD found in
org/ephman/abra/tools/codegen.dtd. (Also listed in the Appendix)

\subsection{Basic Class Description}
Our example will walk through creation of a Class User, and building
upon each step we can go from a simple class description to generating
a complex data storage persistence layer.  The first entry to create a user goes as follows
\begin{verbatim}
  <class name="org.ephman.examples.user.generated.User">
    <description>This class represents a User</description>
    <field name="userId" type="string" required="true"/>
    <field name="userName" type="string"/>
  </class>
\end{verbatim}
This describes a class and it's fields\footnote{A full description of
the dated is in Appendix A}.  Running MapToJava on this XML
file will produce a java file in
org/ephman/examples/user/generated/ called
User.java.\footnote{The default output directory is . to output to a
different base path then the working Dir use -outdir switch (see
appendix A)}  This
class will have two protected member variables \_userId and
\_userName, both of type java.lang.String\footnote{see Appendix C for
a list of supported XML fields and their Java/SQL mappings}. Also public routines
\begin{verbatim}
  public String getUserId ()
  public void setUserId (String foo)
  public String getUserName ()
  public void setUserName (String foo)
\end{verbatim}
will be generated for get/set operations on these fields.

Also note the 'description' tag this text is placed at the top of the
class (good for Javadoc), and \emph{required="true"} was placed on the
userId field.  This won't generate any extra Java code but is a
directive to the XML Marshaller (and SQL, Validation) that this field must exist.

\section{Storing in the Database}
\subsection{Simple storage}
Okay, so we have a class \texttt{User} but now we want to get it stored in the
database.  The first step for all our database classes is to add a
field called 'oid' of type integer (required by Abra) and specify that
it is the identity column.

This would look something like this.
\begin{verbatim}
<!-- property nodes can be used to define common variables like
packages and regular expressions -->
  <property name="user.pkg" value="org.ephman.examples.user.generated"/>

  <class name="${user.pkg}.User"
          identity="oid">
    <description>This class represents a User</description>
    <field name="oid" type="integer"/>
    ...
  </class>
\end{verbatim}
This piece of code issues the directive that the primary key will be
'oid'(\texttt{identity} statement).


Next we need to describe the database schema of the class \texttt{User}.  This
consists of two parts:  putting a \textit{map-to} element inside the class,
which gives the name of the table that user should be stored in
and assigning column names to each field using the \textit{sql} element.
\begin{verbatim}
  <class name="${user.pkg}.User"
          identity="oid">
    <description>This class represents a User</description>
    <map-to table="example_user"/>

    <field name="oid" type="integer">
      <sql name="user_oid"/>
    </field>
    <field name="userId" type="string" len="32">
      <sql name="user_id"/>
    </field>
    <field name="userName" type="string" len="80">
      <sql name="user_name"/>
    </field>
  </class>
\end{verbatim}
This defines the database table to be \textit{example\_user} and maps each field to
their column(so the Java field userName maps to USER\_NAME in the DB).
\marginpar[left]{this will also affect later validation}
Notice that the 'len' attribute has been added to specify how long
each string should be. 

\subsection{Naming Conventions}
At this point running MapToJava with the options \textsf{-factories} and
'-schema \textless file-name\textgreater'  will create the usual java files and in the
same packages as each file create a Factory.  The naming convention is
Abstract\textless file-name\textgreater Factory.java : so User.java
would be accompanied by AbstractUserFactory.java.

The abstract factory contains a listing of all the java/sql field
mappings, the table name, and pre-coded insert, update, and delete strings.
Abstract routines from FactoryBase like (getTableName(), etc.) are
implemented.\footnote{a sample of the actual AbstractUserFactory is in
Appendix D}  Also found are the names of the generated stored
procedure name and how to call it (if the '-procs' switch was used in
generation).  The methods getByOid, and store are also implmented.


The next step is the declared factory.  One needs to define a factory
(usually \textless class-name\textgreater Factory), to extend the
abstract Factory and implement any methods you need.  To UserFactory
we would write any desired queries as follows.
\begin{verbatim}
	public User getUserById (DatabaseSession dbSess, String id) throws
SQLException {
		return (User)super.getObject (dbSess, super.userId, id);
	}
\end{verbatim}
It's that simple.
\subsection{Complex Storage and Heavy Lookups/Storage}
What if the user were to have a reference to a Company object which
represents the company that the user works for.  The standard database
method is to use a 'foreign-key'.  Simply put, if the Company has an
oid of n then a row in the user table for company\_oid's would need to
store n as this users value.


This is handled by the persistence layer for each object which has a
foreign table the primary key (oid) is stored in an
foreign\_oid column as an int.  HOWEVER, standard object lookups are
light - meaning they don't retrieve the foreign objects.  So simple
retrieval of a User his company oid will be set to n but the Company
object will be null.

To handle heavy lookups one must override a routine in the factory
called \emph{deepRetrieval}.  The code in our UserFactory might look
like this.
\begin{verbatim}
...
  protected void deepRetrieval (DatabaseSession dbSess,
                 Identified item) throws SQLException {
    User user = (User)item;
    user.setCompany (CompanyFactory.getInstance ().getByOid (dbSess,
            user.getCompanyOid ());
  }
...
\end{verbatim}
Notice that we call CompanyFactory which you would have to write to extend
the abstract factory and implement the singleton pattern.  Also
note that the only XML entry looked like.
\begin{verbatim}
  <field name="company"
      type="org.ephman.examples.user.generated.Company">
    <sql name="company_oid"/>
  </field>
\end{verbatim}
And from this both a field for company and an int companyOid were placed
in the Java file and only 'int company\_oid' was created in schema.


For storage purposes there are two routines, deepStorage and
preStorage.  'deepStorage' handles things like Vectors where each
entry has an oid column referencing back to this object, and thus
needs this object to be stored first.  'preStorage' is for Objects
that the current one depends on: for a User to store his companies oid in
the db, the Company object must be stored first, so the 'preStorage'
routine allows you to save any objects which are referenced in a
single-entry in this object.
\subsection{Many 2 Many}
Handling many to many relationships between two classes(each with its own
table) requires a third table to store the relationship.  The code
generator will handle this and provide strings to handle inserting a
relationship as well as lookups each way(So for User to Group a lookup
getting all the Users in a Group, or all the Groups for a given User).
\begin{verbatim}
 <class name="${user.pkg}.UserToGroup" 
      many-to-many="true" 
      factory-ext="ManyToManyFactoryBase"
      >
    <map-to table="merlin_user_desk"/>

  <field name="user" type="${user.pkg}.User"
     >
    <sql name="user_oid"/>
  </field>
  <field name="desk" type="${user.pkg}.Group">
	<sql name="desk_oid"/>
  </field>
 </class>
\end{verbatim}


Note: the field many-to-many was set to true : and the base factory
was set to ManyToManyFactoryBase.

\subsection{Adding constraints}
Now that we have a class, which is stored in the DB we might like
to add a field and constrain it to certain values.  An example would
be if we added a field called age which was an int, this field must be
\textgreater = 0 so an entry to place this constraint looks like this:
\begin{verbatim}
  <field name="age" type="integer">
    <sql name="user_age" constraint-name="user_age_gte_zero"
        constraint=">= 0"/>
  </field>
\end{verbatim}
This code will add a constraint on age such that an exception is
thrown and the operation not performed if someone tries to set an age
less than zero.
\paragraph{Referential constraints} are used to guarantee that if a User works
at ``company\_oid=7'' that in the Company table there is a row which
has ``oid=7''.  These are automatically added on schema generation,
however the names default to a mix of the class and field name.
Instead one can give better names by using the 'constraint-name'
attribute, but no entry is required in the 'constraint' attribute.

\subsection{End-dating}

Some objects like users, are referenced all over the database, and
when a user is removed from the system it is important not to delete
them.  Instead we must keep the user (in order for those things that
reference him to still be valid), and end-date them.  This is done by
adding a time-stamp to an end-date column which represents the time at
which the user was removed from the system.

To do this you simply add an attribute to the \textit{class} element.

\begin{verbatim}
<class name="org.ephman.examples.user.generated.User"
      identity="oid" end-date="true">
  ... 
\end{verbatim}

The \textit{end-date="true"} will specify to have an end-date column -
and the extended factory now becomes \emph{EndDateFactoryBase} which
extends FactoryBase and has extra code which delimits searches for
end-date=null, and when \emph{delete} is called the object is
end-dated(not removed).

\subsection{In-lining}
Suppose we wished to add an Address to a User.  But a type like
Address has a one to one relationship with User and there is no need
to store it in a separate table in the database, the directive
\textit{inline=''true''} will define the columns of address inside of
the user table.  The \textit{prefix} tag allows an optional prefix to
be concatenated to the columns defined in the inline-type
\begin{verbatim}
<class name="${user.pkg}.Address"
      identity="oid">
   <!-- oid is defined but not used when inlined -->
    <field name="oid" type="integer">
       <sql name="ad_oid" />
    </field>
	<field name="street" type="string" len="64">
      <sql name="ad_street" />
    </field>    
    <field name="city" type="string" len="20">
      <sql name="ad_city" />
    </field>
    <field name="state" type="string" len="2">
      <sql name="ad_state" />
    </field>
</class>
\end{verbatim}
\marginpar{the prefix allows multiple in-lines of the same type}
To include some address in class user the syntax would be.
\begin{verbatim}
<class name="${user.pkg}.User"
      identity="oid" end-date="true">
  ... 
   <field name="homeAddress" type="${user.pkg}.Address" >
      <sql name="ad" inline="true" prefix="home" />
    </field>
   <field name="workAddress" type="${user.pkg}.Address" >
      <sql name="ad" inline="true" prefix="work" />
    </field>
...
\end{verbatim}

\section{Abstracting data views}
In order to allow changes to database code and classes without messing
up any published API and our internal components (Web/Fix/Swift/..) that send
messages back and forth to a central server, we create views.  These views are
simple data extractions of the classes they represent containing any
subset of the fields defined in the XML.
\subsection{Adding a view}
Suppose the class User has grown to 40 different fields, but we want a
lookup of several users to pass to someone(maybe the list of users at
a company).  Now what if all that was needed was the name, Id and
email of each user.  If the list is hundreds long and we are
Serializing (or marshaling to XML) all 40 fields for each user, we
are seriously hurting performance.


The solution is to pass only the fields we need in Views of the User.
to do so we do the following

\begin{verbatim}
<property name="views.pkg" value="org.ephman.examples.user.views"/>

<class name="${user.pkg}.User" ... >

    <class-view name="${views.pkg}.UserQuickView"
format="quick" />
...
    <field name="userId" type="string">
      <view in="quick"/>
    </field>
...
</class>
\end{verbatim}
This will generate a class representing an abstraction of the class
User, which has the field \emph{userId}.  ALSO a routine
will be added to User called \textit{createQuickView ()}. This routine
operates on \textit{this} and returns a \textit{UserQuickView} with
userId set to the same value in the User object.  Also you can create
a new User with a UserQuickView as the initialization parameters
(ie. from a new User form).
\begin{verbatim}
UserQuickView initUser -> passed in from a separate service..
User newUser = new User ();
newUser.initializeFromView (initUser);
\end{verbatim}
\subsection{Composite views}

A composite field may be added to a view in two ways.
\begin{itemize}
\item A single field is needed i.e. a user has a company
that is of type Company - but in a blotter view the only field needed is the
\textit{companyName}.  Thus, in the view we specify the
\textit{foreign-field} attribute:
\begin{verbatim}
    <field name="company" type="${user.pkg}.Company">
      <view in="quick" foreign-field="companyName"/>
    </field>
...
\end{verbatim}
note: that \textit{companyName} needs to be a valid field in the
Company object.
\item More than one field is needed - so a view of the composite field
is required.   The \textit{foreign-view} attribute is used instead and
a view that exists in the other class must be specified.  \marginpar{this implies that the view 'simple' has been defined in the
Company class}

\begin{verbatim}
    <field name="company" type="${user.pkg}.Company">
      <view in="quick" foreign-view="simple"/>
    </field>
...
\end{verbatim}

\end{itemize}

\section{Validation}
Another crucial part of designing a large system is the ability to
validate your data classes.  For simple validations like mandatory
fields, string length, and regular expression matching, Abra allows
 \textit{validate} nodes for each field.  Combined with the generation
switch '-validation' Abra will generate classes of the name
Base\textless class-name\textgreater Validator in the same package as
the class is 
generated.  


For 'deepValidation' - that is validating composite sub-objects and for
complicated multi-field, or DB lookups these must be written by hand.


\subsection{Mandatory, Length}
Fields can be defined mandatory at two levels.  On the 'field' node the
tag \textit{required=true}, this will add a 'NOT NULL' directive do
the database schema and assert that this field exists on validation.
If the field is required for business reasons, but needs to be stored
in any case the \textit{required=true} tag can be added in the
'validate' node.
\begin{verbatim}
...
<field name="workAddress" >
   <sql name="w_add" inline="true" prefix="work"/>
   <validate required="true" />
</field>
...
\end{verbatim}
Length validations are generated automatically for each string using
the \textit{len=''\textless size\textgreater ''} tag.

\subsection{ValidationExceptions}
When a validator encounters a failed assertion it continues processing
that object, then throws a validation exception containing a Vector of
FieldErrors.  The method getErrorCode will return an int which
defaults to the values in abra.validation.ValidationCodes
(LENGTH\_FAILED, MANDATORY\_FAILED, REGEXP\_FAILED).  \marginpar[left]{If
the field is a String the error will be StringError} For the regular
expressions this can be overriden by using the \textit{code=} tag and
a new error name can be generated by using the \textit{name=} tag.

\section{XML}

All generated classes have canonical XML representations and can be
marshaled to and from XML using the class abra.tools.Marshaller.  The
map file(s) can be added to a new instance of the marshaller using the
\textit{addMapFile} call.

\appendix
\chapter {Full listings of Switches and Options}

\section{XML Map file options}
These can also be found in codegen.dtd at http://charzard.


\# note include and relation are to be implemented ..

\begin{itemize}
\item{mapping ( class+,  property* )}
the top level element for all Abra files

\item{class ( description?, cache-type?, map-to?, class-view+, constraint+, field+ )}
\begin{description}
\item[name] the fully quaified class name - (ie. com.foo.Bar)
\item[extends] a Java class to extend (does not have to be in the map-file)
\item[identity] for SQL-mapping the primary key of this table
\item[implements] any Java interfaces to implement (; seperated) 
\item[factory-ext] the Java base factory for the generated Abstract fac defaults to (FactoryBase or EndDateFactoryBase (end-date=''true'')
\item[many-to-many] set to true for many to many relationships
\item[end-date] set to true if the class should be end-dated not deleted.
\end{description}

\item{constraint}
\begin{description}
     \item[name] 
     \item[type] 'unique' or 'check'
\end{description}
\item{relation ( description?, map-to?, related, related) } : NOT implemented
      name        
\item{related ( description?, sql?, xml?, ldap? )} : NOT implemented
\begin{description}
    \item[name]
    \item[type] 
\end{description}
\item{ map-to}
\begin{description}
    \item[table] the sql table name
    \item[xml] the xml node name
\end{description}

\item{class-view}: a definition of a class data-view (to abstract this
class)
\begin{description}
    \item[name]
    \item[format] a short name to be use later like 'normal', 'blotter'
\end{description}



\item{field ( description?, sql?, xml?, view+ , validate*)} : description of a java/sql/xml field mapping 
\begin{description}
    \item[name]
    \item[type] Fully qualified name (integer, string or foo.bar.FooBar)
    \item[required] if true both sql and xml will require this field
    \item[collection]    ( array | vector | hashtable | collection | set | map ) 
\end{description}

\item{sql}: for field to sql mappings
\begin{description}
    \item[name]
    \item[constraint-name] for referential or defined contraints
    \item[constraint] to define a field constraint (like '> 0')
    \item[type] to re-define the sql type (otherwise the default
mappings are used
\end{description}

\item{view}: to define how this field is viewed in different data abstractions
\begin{description}
    \item[in] the format of the view (like 'blotter')
    \item[foreign-view] if a composite field the name of the view in
the other class
	\item[foreign-field] if a composite field the single parameter to rip
\end{description}
 
\item{xml}: to define a how the java maps to xml
\begin{description}
    \item[name]
    \item[type]
    \item[node]  ( attribute | element | text ) how it should be (un)/marshalled
\end{description}
\item{validate}: add a validation node to this field
\begin{description}
	\item[required] set to true for business level mandatory field
	\item[regex] give a regular expression for pattern matching
	\item[name] used for regexp failed messages
	\item[code] used for regexp failed codes
\end{description}
\end{itemize}


\section{Switches in MapToJava}
The list of switches available in MapToJava. 

\begin{itemize}
\item Generation options
\begin{description}
\item{-outdir \textless directory\textgreater} : redifines the base output directory from . to
the given one

\item{-schema \textless schema-file\textgreater} : create a schema file of this name

\item{-factories} : create factories

\item{-facext \textless factory-name\textgreater} : if -factories is
set use this factory as base (unless a class explicitly declares
factory-ext=) (DEFAULTS to FactoryBase)
\item{-implements \textless class-name\textgreater} : use this class
as a global implements for all classes generated.
\item{-verify} : generate nothing just make sure the map is valid XML
\item{-validation} : generate Base Validators
\item{-procs} \textless pkg\textgreater : generate stored procedures and use them to store and update
\item{-xp} : use Abra Xml Parser instead of Xerces
\end{description}
\item Text Feedback options

\begin{description}
\item{-supress} : only print out Error messages
\item{-verbose} : a longer print out listing each class touched
\item{-mega} : a mega print out with all the verbose info and info
about each member field.
\end{description}

\item Props file
\begin{description}
\item {-props \textless props-file\textgreater} : instead of the above
command-line flags a props file can be specified with all the above
flags.
\item an example file
\begin{verbatim}
outdir=/home/paulb/Development/java
files=/home/paulb/Development/schema/test_map.xml
schema=../schema/new_merlin_schema.sql
factories=true
\end{verbatim}

\end{description}

\end{itemize}
\section{List of xml to Java/SQL mappings}


\begin{tabular}{|c|c|c|c|}
\hline
XML & Java & SQL & notes \\ \hline \hline
string & java.lang.String & varchar (n) & n is specified in the len
attribute \\ \hline

integer & int & int & - \\ \hline

date & java.sql.Timestamp & date & db uses GMT.. \\ \hline

long & long & number (38,0) & - \\ \hline

boolean & boolean & varchar (1) & the 'T'/'F' is stored and retrieved \\

float & float & float & - \\ \hline

double & double & double & - \\ \hline

longstring & java.lang.String & clob & when a string could be \textgreater then 4000 characters \\ \hline

\hline
\end{tabular}

note: to create a vector use the following syntax

\begin{verbatim}
<field name="children" type="com.foo.Person" collection="vector"/>
\end{verbatim}
And for an array use \textit{collection=''array''}
The type refers to the type of each element and the
collection="vector" makes this a vector.



\end{document}

 